\documentclass[licencjacka]{pracamgr_2}
%\usepackage{latexsym}
\usepackage[MeX]{polski}
\usepackage[utf8]{inputenc}
\usepackage{graphicx}
\usepackage{afterpage}
\usepackage{rotating}
%\usepackage{subfigure}
\usepackage{polski}
\usepackage{natbib}

\author{ Tu wpisz imię i nazwisko autora }
\nralbumu{ tu wpisz numer albumu }
\title{ Tytuł pracy w języku polskim.}
\tytulang{ Tytuł pracy w języku angielskim}
\kierunek{Zastosowania Fizyki w Biologii i Medycynie}
\opiekun{stopień Imię i Nazwisko promotora \\ Zakład Fizyki Biomedycznej \\ Instytut Fizyki Doświadczalnej}
\dziedzina{13.200}
\specjalnosc{Specjalność}
\date{Miesiąc i rok złożenia pracy}
\keywords{słowa kluczowe}
\bibliographystyle{elsart-harv}

\begin{document}
\maketitle
\let\cleardoublepage\clearpage

\begin{abstract}
Tu wpisujemy streszczenie
\end{abstract}
\tableofcontents
\chapter*{Cel pracy}
\addcontentsline{toc}{chapter}{Cel pracy}
Cel pracy.
\chapter{Wstęp}
Tu piszemy informacje wprowadzające w tematykę  pracy, potrzebne do zrozumienia treści.
\section{Pierwsza sekcja}
Przykładowa cytacja \citet{Shepherd}.
\chapter{Dane eksperymentalne}

\chapter{Metodologia}
\chapter{Wyniki}
\chapter{Dyskusja}
\chapter{Podsumowanie}
\bibliography{bibliografia}
\end{document}
